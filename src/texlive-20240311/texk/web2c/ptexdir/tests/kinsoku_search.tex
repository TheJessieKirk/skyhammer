%#!euptex -ini -etex
\tracingonline1
\tracingassigns1

% 最優先hashはコードポイントから一意に決まる値(jtex.pdfも参照)
\prebreakpenalty`、=10000 % code=12289, 最優先hash=1
\prebreakpenalty`。=10000 % code=12290, 最優先hash=2
\prebreakpenalty`ぁ=150 % code=12353, 最優先hash=1だが埋まってしまったので3に分配
\prebreakpenalty`ぃ=150 % code=12355, 最優先hash=3だが埋まってしまったので4に分配

\showthe\currentgrouplevel % > 0.
\showthe\prebreakpenalty`ぃ % > 150.

\prebreakpenalty`ぃ=0 % changing hash=4
\showthe\prebreakpenalty`ぃ % > 0. ← OK(hash=4読出)

\prebreakpenalty`ぃ=123 % changing hash=4
\showthe\prebreakpenalty`ぃ % > 123. ← OK(hash=4読出)

\prebreakpenalty`ぁ=0 % changing hash=3
\showthe\prebreakpenalty`ぃ % > 123. ← OK(hash=4読出)

% この時点で "ぁ" が消えてhash=3が空いている。
% 一方,"ぃ"はフォールバックしたhash=4に格納されている。

\prebreakpenalty`ぃ=0 % changing hash=4
\showthe\prebreakpenalty`ぃ % > 0. ← 修正(削除済み)

\prebreakpenalty`ぃ=234 % changing hash=3
\showthe\prebreakpenalty`ぃ % > 234. ← OK(hash=3読出)

\prebreakpenalty`ぃ=0 % changing hash=3
\showthe\prebreakpenalty`ぃ % > 0. ← 修正(削除済み)

\prebreakpenalty`ぁ=345 % changing hash=3
\showthe\prebreakpenalty`ぃ % > 0. ← 修正(削除済み)

% この時点で "ぁ" が再びhash=3を埋めている。

\prebreakpenalty`ぃ=0 % reassigning hash=4
\showthe\prebreakpenalty`ぃ % > 0. ← OK(削除済み)

\end
