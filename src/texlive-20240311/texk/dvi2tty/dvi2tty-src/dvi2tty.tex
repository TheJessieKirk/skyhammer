%
%  By: Robert Schneider
%
%	LaTeX input file which describes how dvi2tty works under
%	VMS with VAXC.  When using this file as a source for documentation
%	of dvi2tty on your system,  look for the string "NOTE ---" to
%	see what local variables need to be set.
%
%	This file was created by modifying the original Unix man page.
%	The majority of the conversion from troff to TeX was done by
%	tr2tex. 
%
\documentstyle{article}
%
%	Set the page margins.
%
\parindent 0pt
\marginparpush 5pt 
\headheight 5pt
\headsep 0pt
\footheight 5pt
\footskip 15pt 
\columnsep 10pt
\columnseprule 0pt 
\marginparwidth 0pt
\oddsidemargin  0pt
\evensidemargin  0pt
\marginparsep 0pt
\topmargin   0pt
\textwidth   6.5in
\textheight  9.0in
%
\newcommand{\bs}{$\backslash$}
%
\title{{\bf DVI2TTY for VAX/VMS}}
\author{}
\date{April 25, 1990} 
%
\begin{document}
\maketitle
%
%	NOTE --- We like to describe what something is and how to get it
%		 at the top of our files.  You will probably want to
%		 delete from here to the next \section.
%
This document describes the use of {\bf dvi2tty}\/, a filter for converting
the device independent output files created by \TeX\/ to files which may
be printed on any ASCII terminal or line printer.
This document may be printed on the line printer in CPE-3.102 by entering
the command;
\\[1ex]
{\em DOCPRINT SYS\$DOCUMENT:DVI2TTY.DOC}
\\[1ex]
Copies of this document may be printed on the laser printer in CPE-3.159
by entering the command;
\\[1ex]
{\em PSPRINT SYS\$DOCUMENT:DVI2TTY.PS}
\\[1ex]
You may view copies of this document on workstations or personal computers
which recognize the {\it X}\/ window system protocol by entering the command;
\\[1ex]                                                          
{\em XDVI SYS\$DOCUMENT:DVI2TTY.DVI}
\section*{Description}
{\bf dvi2tty}
converts a \TeX\/ DVI-file to a format that is appropriate for terminals
and line printers. The program is intended to be used for
preliminary proofreading of \TeX-ed documents.
By default the output is directed to the terminal.
%
%	NOTE --- VMS can't pipe output to a pager.  Maybe you are lucky
%		 and your system can.
%
%,possibly through a pager (depending on how the program was installed),
%but it can be directed to a file or a pipe.
\par\vspace{1.0\baselineskip}
The output leaves much to be desired, but is still
useful if you want to avoid walking to the
laser printer (or whatever) for each iteration of your
document.
\par\vspace{1.0\baselineskip}
Since {\bf dvi2tty}\/ produces output for terminals and line printers the
representation of documents is naturally quite primitive.
Font changes are totally ignored, which implies that
special symbols, such as mathematical symbols, get mapped into the 
characters at the corresponding positions in the ``standard'' fonts.
\par\vspace{1.0\baselineskip}
If the width of the output text requires more columns than fits
in one line (c.f. the -w option) it is broken into several lines by
{\bf dvi2tty} although they will be printed as one line on regular \TeX\/ output
devices (e.g. laser printers). To show that a broken line is really
just one logical line an asterisk (``*'') in the last position
means that the logical line is continued on the next physical
line output by {\bf dvi2tty.}
Such a continuation line is started with a space and an asterisk
in the first two columns.
\section*{Format}
{\bf dvi2tty}
[
{\bf -o}
{\it file}
]
[
{\bf -p}
{\it list}
]
[
{\bf -P}
{\it list}
]
[
{\bf -w}
{\it n}
]
[
{\bf -l}
]
[
{\bf -u}
]
[
{\bf -s}
]
{\it file\rm[\it.dvi\rm]}
\par\vspace{1.0\baselineskip}
{\bf dvi2tty}\/ is compiled with VAXC which converts uppercase command line
arguments to lowercase.  To pass an uppercase argument to {\bf dvi2tty},
the argument must be enclosed in quotation marks ( i.e. ``-P'' ).  Arguments
enclosed in brackets ( [ ] ) are optional.
\section*{Options}
The {\it file}\/ [{\it .dvi}] argument is optional.
Options may be specified in either a symbol or logical which is named
%
%	NOTE --- The definition of the following environment variable
%		 is implementation dependent.
%
DVI\$DVI2TTY or on the command line.
Any option on the command line, conflicting with one in the
environment, will override the one from the environment.
\begin{list}{}{\setlength{\leftmargin}{0.6in}\labelwidth 0.6in}
\item[{{\bf -o} {\it name}}]
Write output to file {\it name}.
\item[{{\bf -p} {\it list}}]
Print the pages chosen by {\it list}.
Numbers refer to \TeX-page numbers (known as \bs count0).
An example of format for {\it list}\/ is ``1,3:6,8''
to choose pages 1, 3 through 6 and 8.
Negative numbers can be used exactly as in \TeX,
e g -1 comes before -4 as in ``-p-1:-4,17''.
\item[{{\bf -P} {\it list}}]
Like -p except that page numbers refer to
the sequential ordering of the pages in the dvi-file.
Negative numbers don't make a lot of sense here...
\item[{{\bf -w} {\it n}}]
Specify terminal width {\it n.}\/ The legal range is 16-132 with a
default of 80. If your terminal has the
ability to display in 132 columns it might
be a good idea to use -w132 and toggle the
terminal into this mode as output will probably look somewhat better.
%
%	NOTE --- The ability to pipe to a pager is operating system dependent.
%
%\item[{{\bf -q}}]
%Don't pipe the output through a pager.
%This may be the default on some systems
%(depending on the whims of the SA installing the program).
%\item[{{\bf -f}}]
%Pipe through a pager, \$PAGER if defined, or whatever your SA compiled
%in (often ``more''). This may be the default, but it is still okay
%to redirect output with ``$>$'', the pager will not be used if output
%is not going to a terminal.
%\item[{{\bf -F}}]
%Specify the pager program to be used.
%This overides the \$PAGER and the default pager.
%\item[{{\bf -Fprog}}]
%Use ``prog'' as program to pipe output into. Can be used to choose an
%alternate pager (e g ``-Fless'').
\item[{{\bf -l}}]
Mark page breaks with the two-character sequence ``\^{}L''. The default is
to mark them with a form feed character.
\item[{{\bf -a}}]
Remove accents grave etc. from output.
\item[{{\bf -c}}]
Do not attempt to trnaslate any characters (like the Scandinavion mode)
except when running in tt-font.
\item[{{\bf -C}}]
Do not attempt to compose a combining character sequence.
\item[{{\bf -bdelim}}]
Print the name of fonts when switching to it (and ending it). The delim
argument is used to delimit the fontname.
\item[{{\bf -u}}]
Toggle option to process certain latin1 characters. Use this if your output
devices supports latin1 cahracters.
Note this may interfere with -u. Best not to use -u and -s together.
\item[{{\bf -s}}]
Toggle option to process the special Scandinavian characters that on most (?)
terminals in Scandinavia are mapped to ``\{$|$\}[\bs ]''.
Note this may interfere with -u. Best not to use -u and -s together.
\item[{{\bf -J}}]
Auto detect NTT JTeX, ASCII pTeX, and upTeX dvi format.
\item[{{\bf -N}}]
Display NTT JTeX dvi.
\item[{{\bf -A}}]
Display ASCII pTeX dvi.
\item[{{\bf -U}}]
Display upTeX dvi.
\item[{{\bf -Eenc}}]
Set output multibyte encoding. The enc argument 'e', 's', 'j', 'u', and 'u1'
denotes EUC-JP, Shift\_JIS, ISO-2022-JP, UTF-8,
and UTF-8 (do not use ligature for ff, fi, fl, ffi, ffl), respectively.
\end{list}
%
%	NOTE --- No other files are used under VMS.
%
%\section*{FILES}
%/usr/ucb/more \ \ \ \ 
%probably the default pager.
\section*{Environment}
%
%	NOTE --- The PAGER variable isn't used under VMS.
%
%PAGER \ \ \ \ \ \ \ \ \ \ \ \ 
%the pager to use.
%\nwl
%
%	NOTE --- The definition of the following environment variable
%		 is operating system dependent.
%
Either a symbol or logical which is named DVI\$DVI2TTY can be set to hold
command line options.  As an example,  if you wish to normally have your
output written to the file {\it dvi2tty.output}\/ and wish to also have
the output be 132 columns in width,  you may define the logical name
DVI\$DVI2TTY as in;
\par\vspace{1.0\baselineskip}
\begin{center}
{\bf \$ DEFINE DVI\$DVI2TTY ``{\it o dvi2tty.output w 132}''}
\end{center}
\par\vspace{1.0\baselineskip}
Options set in the environment variable will be overridden by conflicting
options on the command line.
\section*{See Also}
%
%	NOTE --- How do you describe where your help files and manuals are?
%
For further information use the DCL command {\bf HELP TeX}\/ to obtain
information on \TeX\/ and how \TeX\/ may be used on our system.
You should also read the \LaTeX\/ {\it Local Guide}\/ which may be found
in SYS\$DOCUMENT:.
This document may be printed on the line printer in CPE-3.102 by entering
the command;
\\[1ex]
{\em DOCPRINT SYS\$DOCUMENT:LOCAL\_GUIDE.DOC}
\\[1ex]
Copies of this document may be printed on the laser printer in CPE-3.159
by entering the command;
\\[1ex]
{\em PSPRINT SYS\$DOCUMENT:LOCAL\_GUIDE.PS}
\\[1ex]
You may view copies of this document on workstations or personal computers
which recognize the {\it X}\/ window system protocol by entering the command;
\\[1ex]
{\em XDVI SYS\$DOCUMENT:LOCAL\_GUIDE.DVI}
\section*{Bugs}
Blanks between words get lost quite easy. This is less
likely if you are using a wider output than the default 80.
\par\vspace{1.0\baselineskip}
Only one file may be specified on the command line.
\section*{Author}
Svante Lindahl, Royal Institute of Technology, Stockholm
\\[1ex]
\{seismo, mcvax\}!enea!ttds!zap
\\[1ex]
Improved C version: Marcel Mol
\\[1ex]
marcel@mesa.nl, MESA Consulting
\end{document}

